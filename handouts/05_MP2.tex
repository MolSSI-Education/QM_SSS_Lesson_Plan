%\documentclass[12pt]{article}
\documentclass[aip,jcp,preprint,superscriptaddress,floatfix]{revtex4-1}
\usepackage{url,graphicx,tabularx,array,geometry,amsmath,listings}
\setlength{\parskip}{2ex} %--skip lines between paragraphs
\setlength{\parindent}{20pt} %--don't indent paragraphs

\setlength{\headheight}{-50pt}
\setlength{\textheight}{700pt}
\setlength{\textwidth}{500pt}
\setlength{\oddsidemargin}{-10pt}
\setlength{\footskip}{50pt}
\usepackage{graphicx}% Include figure files
\usepackage{bm}
\graphicspath{{./Figures/}}

%-- Commands for header
\renewcommand{\title}[1]{\textbf{\large{#1}}\\}
\renewcommand{\line}{\begin{tabularx}{\textwidth}{X>{\raggedleft}X}\hline\\\end{tabularx}\\[-0.5cm]}
\newcommand{\leftright}[2]{\begin{tabularx}{\textwidth}{X>{\raggedleft}X}#1%
& #2\\\end{tabularx}\\[-1cm]}

%\linespread{2} %-- Uncomment for Double Space
\begin{document}

\title{\center{Second-order M\o{}ller-Plesset Perturbation Theory (MP2)}}
\rule{\textwidth}{1pt}
\leftright{The Molecular Sciences Software Institute}{Daniel G. A. Smith} %-- left and right positions in the header

\bigskip
\section{Canonical MP2}
For a canonical RHF reference the MP2 equations are rather straightforwardly,

\begin{equation} \label{eq:mp2} E_{\mathrm{MP2}} = \frac{ (ia|jb) (ia|jb) } {
\epsilon_{i} + \epsilon_{j} - \epsilon_{a} - \epsilon_{b} } - \frac{ [(ia|jb) - (ib|ja)]
(ia|jb) } { \epsilon_{i} + \epsilon_{j} - \epsilon_{a} - \epsilon_{b} }
\end{equation}

Where the first term is often noted as the opposite-spin (OS) term while the second, the same-spin (SS) term.
Its worth thinking about why these are labeled in this way, and why the same-spin term has an exchange-like component to it.
There is also a variant of MP2 which attempts to correct for the MP2 errors by scaling the spin term separately, this is denoted as spin-component scaled MP2 (SCS-MP2)\cite{Grimme:2003:9095}.
The following parameters are often used,
\begin{equation}
E_{SCS-MP2} = \frac{1}{3} E_{MP2-SS}  +\frac{6}{5}  E_{MP2-OS}
\end{equation}

The rate-limiting step of MP2 is the four-index transformation of the ERI tensor from atomic to molecular orbitals,
\begin{equation}
(ia|jb) = C_{\mu i} C_{\nu a} (\mu\nu|\lambda\sigma) C_{\lambda j} C_{\sigma b}
\end{equation}

If this is performed in a single step this cost is  $\mathcal{O}$N$^8$! However, factoring this However, factoring this transformation leads to rather straightforward intermediates and the overall contraction scales as $\mathcal{O}$N$^5$ as seen below,

\begin{equation}
\begin{split}
(i\nu|\lambda\sigma) &\leftarrow C_{\mu i} (\mu \nu | \lambda \sigma) \\
(i\nu|j\sigma) &\leftarrow C_{\lambda j} (i \nu | \lambda \sigma) \\
(ia|j\sigma) &\leftarrow C_{\nu a} (i \nu | j \sigma) \\
(ia|jb) &\leftarrow C_{\sigma b} (ia|j \sigma) \\
\end{split}
\end{equation}

Notice the order of occupied indices and then virtual indices, as the virtual index is generally much larger than the occupied one this ordering is quite important for performance.

\section{Density-Fitted MP2}

The density-fitted discussion in the JK algorithm notes recall,
\begin{equation}
\label{eq:dferi}
g_{\mu \nu \lambda \sigma} \approx (\mu \nu |P) (P| \lambda \sigma)
\end{equation}

At this point is would be trivial to reform the $(ia|jb)$ tensor; while this would be slightly cheaper than the
direct transformation of the 4-index integral above we would still create a large matrix in memory which is best avoided. 

A better approach would be to build small blocks of the $(ia|jb)$ tensor at a time. Consider the following pseudocode,

\begin{verbatim}
for i in range(ndocc):
    for j in range(i, ndocc):
          I_ab <- iaQ jbQ
\end{verbatim}

In the above we only build a single block of virtual by virtual integrals which can than be contracted into the energy. This is one of the primary benefits of density-fitted MP2 builds.

Unlike the JK example previous you want to use a different fitting basis, in this case the RIFIT which has been optimized to recover MP2-like energies rather than the JK like properties used previously.

\begin{verbatim}
# Build the complementary RIFIT basis for the aug-cc-pVDZ basis (for example)
psi4.core.BasisSet.build(mol, fitrole="RIFIT", other="aug-cc-pVDZ")
\end{verbatim}

\bibliographystyle{aip.bst}
\bibliography{references,konrad_references,jrncodes,mainbib-ds}

\end{document}
